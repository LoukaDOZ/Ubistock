%default packages
\documentclass[a4paper, 12pt]{report}
\usepackage{fancyhdr}
\pagestyle{fancyplain}
\fancyhf{}
\rfoot{\thepage}
\lfoot{Louka Doz \& Thibault Meynier}


%adds the ability to put images and figures in the document
\usepackage{graphicx}

%sets default font
\usepackage[T1]{fontenc}

%sets document langage to french
\usepackage[french]{babel}

%adds landscape page orientation
\usepackage{lscape}

%adds easier chapter styles customization
\usepackage{titlesec}

%adds ssmal and HUGE font size
\usepackage[12pt]{moresize}

%for custom list style
\usepackage{enumitem}

%adds links and hyperlinks
\usepackage{hyperref}

%for extra colors
\usepackage[dvipsnames]{xcolor}

%for better tables
\usepackage{array}

%for bigger tables
\usepackage{longtable}

\usepackage{helvet}

\usepackage{tabularx}

\renewcommand{\familydefault}{\sfdefault}



%changes default lists style for smaller dashes
\setlist[itemize]{label=--}

%changes the display of chapters, 
%do not displays : "chapter n : name_of_the_chapter"
%displays : "n. name_of_the_chapter" instead
\makeatletter
\def\@makechapterhead#1{%
	\vspace*{0\p@}%
	{	\parindent \z@ \raggedright \normalfont
		\ifnum \c@secnumdepth >\m@ne
		  %\if@mainmatter
		    %\huge\bfseries \@chapapp\space \thechapter
		    \huge\bfseries \thechapter.\space%
		    %\par\nobreak
		    %\vskip 20\p@
		  %\fi
		\fi
		\interlinepenalty\@M
		\huge \bfseries #1\par\nobreak
		\vskip 40\p@
	}}
\makeatother


%adds infos to the document
\title{}

\author{
	Louka Doz
	\and 
	Thibault Meynier
}
\date{\raggedleft \today}


\hypersetup{
	colorlinks=true,
	linkcolor={BlueViolet},
	filecolor=magenta,
	urlcolor={MidnightBlue}
}


%starts to write the document
\begin{document}
	\makeatletter
	\begin{titlepage}

		\begin{flushleft}
			\begin{minipage}{4cm}
				\includegraphics[height=4cm]{img/logo_iut}
			\end{minipage}
			\hfill
			\begin{minipage}{5cm}
				\begin{flushright}
		        	\small IUT Sénart-Fonatinebleau\\
		        	département informatique\\
		        	Route forestière Hurtault\\
		        	77300 Fontainebleau\\
				\end{flushright}
			\end{minipage}
		\end{flushleft}

		\vfill

		\begin{center}
	        \vspace*{3cm}%
	        {\HUGE Ubistock}\\[0.5cm]
	        {\huge Rapport d'avancement hebdomadaire}\\[0.5cm]
	        {\Large semaine du 13-06-2020 au 20-06-2020}\\[0.5cm]
        	{\large \textit{Année scolaire 2019-2020}}\\[1cm]
	    \end{center}
	    \vfill
        \begin{raggedright}
	        \begin{description}
	        	\item[\large \underline{Tutrice de projet :}] \large \textbf{Régine Laleau}\\[1cm]
	        	\item[\underline{Par :}] 
		        	\begin{itemize}
			        		\item Louka Doz
			        		\item Thibault Meynier
			        \end{itemize}
	        \end{description}
        \end{raggedright}        
	    \let\newpage\relax% Avoid following page break
	\end{titlepage}
	\makeatother


	%prints a formated table of contents
	\tableofcontents

	%jumps a page

	\newpage

	\chapter*{Synthèse}
		Durant cette semaine, l’équipe projet a réalisé de nouveaux scripts back-end afin de faciliter le développement front-end et faciliter la récupération d’informations. Les fonctions front-end permettant de dialoguer avec l’API ont également été terminées. L’équipe a imaginé et mis en place un nouveau système de connexion plus sûr permettant de ne maintenir une session avec un même utilisateur que 2h avant de le déconnecter.



	 \chapter{Nouvelles implémentations}
	 	\section{Base de données}
	 		\begin{itemize}
	 			\item implémentation du nouveau mécanisme de toen de connexion;
				\item création de la table token.

	 		\end{itemize}

		\section{API}
			\begin{itemize}
				\item demander la liste des logs;
				\item demander la liste des groupes de l’utilisateur;
				\item demander la liste des stockages autorisés de l’utilisateur;
				\item demander des informations sur l’entreprise de l’utilisateur.


			\end{itemize}

		\section{Interface}
			\begin{itemize}
				\item page des groupes;
				\item corrections d’erreurs et mise à jour pour plus de clarté;
				\item implémentation d’une partie des requêtes AJAX.


			\end{itemize}

	\chapter{Changements et ajouts}
		\section{Améliorer l’expérience utilisateur}
			Après avoir terminé de réaliser le prototype des pages, beaucoups de petits changements ont eu lieu pour améliorer l’interface utilisateur ainsi que pour résoudre des problèmes :
			\begin{itemize}
				\item limiter le nombre de fenêtres modales;
				\item utiliser les mêmes mots entre les pages plutôt que des synonymes pour plus de cohérence;
				\item corrections d’éléments qui n’étaient pas responsive;
				\item affichages des droits des utilisateurs en fonction du niveau d’accréditation plus visuel afin de bien montrer la différence entre deux niveaux d’accréditation.

			\end{itemize}

		\section{Token de connexion}
			Désormais, lorsqu’un utilisateur se connecte, un token de connexion est créé contenant son company\_user\_id, un id unique de 32 caractères de long ainsi qu’une date de création. Pour effectuer une action, l’utilisateur envoie maintenant le token\_id et son company\_user\_id pour s’identifier. Si le token est trop vieux (créé il y a plus de 2 heures), la connexion est refusée et l’utilisateur doit se reconnecter. Lors de chaque connexion, les tokens d’il y a plus de 2 heures sont supprimés de la base de données.

		\section{Report du support MySQL}
			L’équipe a finalement décidé de ne pas supporter les bases de données MySQL pour le moment car l’implémentation des fonctionnalités permettant une telle compatibilité ralentit considérablement le développement de l’application.

		\section{Fuseion du travail}
			En fusionnant le travail des deux membres de l’équipe (interface et requêtes vers l’API), le groupe a rencontré de nombreux problèmes, forçant à faire des modifications sur leur serveur local respectif et à modifier et ajouter de nouvelles fonctions réalisant des requêtes AJAX. Régler ces problèmes à pris plus de temps que prévu ce qui a retardé l’implémentation des requêtes AJAX pourtant commencé à temps.


	\chapter{objectifs de la prochaine semaine}
		\begin{itemize}
			\item finir l’ajout des requêtes vers l’API dans les pages existantes;
			\item finir le système de fichiers permettant de gérer les ressources et stockages;
			\item finir le générateur de PDF.

		\end{itemize}
		
\end{document}