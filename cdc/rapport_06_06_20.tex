%default packages
\documentclass[a4paper, 12pt]{report}
\usepackage{fancyhdr}
\pagestyle{fancyplain}
\fancyhf{}
\rfoot{\thepage}
\lfoot{Louka Doz \& Thibault Meynier}


%adds the ability to put images and figures in the document
\usepackage{graphicx}

%sets default font
\usepackage[T1]{fontenc}

%sets document langage to french
\usepackage[french]{babel}

%adds landscape page orientation
\usepackage{lscape}

%adds easier chapter styles customization
\usepackage{titlesec}

%adds ssmal and HUGE font size
\usepackage[12pt]{moresize}

%for custom list style
\usepackage{enumitem}

%adds links and hyperlinks
\usepackage{hyperref}

%for extra colors
\usepackage[dvipsnames]{xcolor}

%for better tables
\usepackage{array}

%for bigger tables
\usepackage{longtable}

\usepackage{helvet}

\usepackage{tabularx}

\renewcommand{\familydefault}{\sfdefault}



%changes default lists style for smaller dashes
\setlist[itemize]{label=--}

%changes the display of chapters, 
%do not displays : "chapter n : name_of_the_chapter"
%displays : "n. name_of_the_chapter" instead
\makeatletter
\def\@makechapterhead#1{%
	\vspace*{0\p@}%
	{	\parindent \z@ \raggedright \normalfont
		\ifnum \c@secnumdepth >\m@ne
		  %\if@mainmatter
		    %\huge\bfseries \@chapapp\space \thechapter
		    \huge\bfseries \thechapter.\space%
		    %\par\nobreak
		    %\vskip 20\p@
		  %\fi
		\fi
		\interlinepenalty\@M
		\huge \bfseries #1\par\nobreak
		\vskip 40\p@
	}}
\makeatother


%adds infos to the document
\title{}

\author{
	Louka Doz
	\and 
	Thibault Meynier
}
\date{\raggedleft \today}


\hypersetup{
	colorlinks=true,
	linkcolor={BlueViolet},
	filecolor=magenta,
	urlcolor={MidnightBlue}
}


%starts to write the document
\begin{document}
	\makeatletter
	\begin{titlepage}

		\begin{flushleft}
			\begin{minipage}{4cm}
				\includegraphics[height=4cm]{img/logo_iut}
			\end{minipage}
			\hfill
			\begin{minipage}{5cm}
				\begin{flushright}
		        	\small IUT Sénart-Fonatinebleau\\
		        	département informatique\\
		        	Route forestière Hurtault\\
		        	77300 Fontainebleau\\
				\end{flushright}
			\end{minipage}
		\end{flushleft}

		\vfill

		\begin{center}
	        \vspace*{3cm}%
	        {\HUGE Ubistock}\\[0.5cm]
	        {\huge Rapport d'avancement hebdomadaire}\\[0.5cm]
	        {\Large \today}\\[0.5cm]
	        {\Large semaine du 30/06/20 au 06/06/20}\\[0.5cm]
        	{\large \textit{Année scolaire 2019-2020}}\\[1cm]
	    \end{center}
	    \vfill
        \begin{raggedright}
	        \begin{description}
	        	\item[\large \underline{Tutrice de projet :}] \large \textbf{Régine Laleau}\\[1cm]
	        	\item[\underline{Par :}] 
		        	\begin{itemize}
			        		\item Louka Doz
			        		\item Thibault Meynier
			        \end{itemize}
	        \end{description}
        \end{raggedright}        
	    \let\newpage\relax% Avoid following page break
	\end{titlepage}
	\makeatother


	%prints a formated table of contents
	\tableofcontents

	%jumps a page

	\newpage

	\chapter*{Synthèse}
		Durant cette semaine, l’équipe projet a réalisé bon nombre de script en vue d’une implémentation graphique future. On peut donc déjà effectuer presque toutes les opérations du cahier des charges via des requêtes à l’api php. L’équipe projet a donc pour objectif de passer à la phase deux de développement, c’est-à-dire la mise en place d’une interface graphique afin d’améliorer l’ergonomie de l’application.


	 \chapter{Nouvelles implémentations}
		\section{Base de données}
			\begin{itemize}
				\item implémentation du nouveau mécanisme des permissions;
				\item implémentation du nouveau système de groupes.
			\end{itemize}

		\section{API}
			\begin{itemize}
				\item création d’entreprises;
				\item ajout/suppression/renommage de stockages;
				\item ajout/suppression/renommage de ressources;
				\item création/suppression d’utilisateurs;
				\item création/suppression de groupes;
				\item assignation/désassignation d’utilisateurs à des groupes;
				\item assignation/désassignation de stockages à des groupes;
				\item déplacement de stockages;
				\item déplacement des ressources;
				\item changement du niveau d’accréditation des utilisateurs;
				\item changement de la quantité d’une ressource;
				\item connexion des utilisateurs.
			\end{itemize}

		\section{Interface}
			\begin{itemize}
				\item page de connexion;
				\item page de création de compte;
				\item page de création d’un nouvel utilisateur;
				\item page de création d’un nouveau groupe.
			\end{itemize}

	\chapter{Changements et ajouts}			
		\section{Nom et prénom des utilisateurs}
			Afin de rajouter un côté professionnel à l’application, le nom d’utilisateur à été remplacés par deux champs : le nom et le prénom de l’utilisateur.

		\section{Email}
			Pour faciliter la connexion des utilisateurs, un nouveau champ pour un email à été ajouté à la base de données. Cet email sera aussi utile à l’avenir, pour contacter un utilisateur de manière personnelle (pour confirmer des actions par exemple).

		\section{Messages de retout de l'API}
			Durant cette semaine, l’équipe s’est accordé sur le retour des messages suite à une requête vers l’API. Les messages retournés seront composés du minimum d’informations nécessaires pour faire tourner et  mettre à jour le client.


		\section{Fonction REGEXP}
    		Dans le cadre du développement de la fonctionnalité permettant de déplacer un stockage dans un autre, l’équipe a souhaité utiliser une fonctionnalité de l’api sqlite permettant de créer ses propres fonctions afin de rendre possible l’utilisation d’expressions régulières dans les requêtes SQL passées à la base de données.

    \chapter{Objectif de la prochaine semaine}
    	\begin{itemize}
    		\item réaliser des scripts permettant la compatibilité avec les bases de données MySQL;
    		\item continuer de développer l'interface;
    		\item terminer l'ensemble des scripts php.
    	\end{itemize}
		
\end{document}