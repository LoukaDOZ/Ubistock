%default packages
\documentclass[a4paper, 12pt]{report}
\usepackage{fancyhdr}
\pagestyle{fancyplain}
\fancyhf{}
\rfoot{\thepage}
\lfoot{Louka Doz \& Thibault Meynier}


%adds the ability to put images and figures in the document
\usepackage{graphicx}

%sets default font
\usepackage[T1]{fontenc}

%sets document langage to french
\usepackage[french]{babel}

%adds landscape page orientation
\usepackage{lscape}

%adds easier chapter styles customization
\usepackage{titlesec}

%adds ssmal and HUGE font size
\usepackage[12pt]{moresize}

%for custom list style
\usepackage{enumitem}

%adds links and hyperlinks
\usepackage{hyperref}

%for extra colors
\usepackage[dvipsnames]{xcolor}

%for better tables
\usepackage{array}

%for bigger tables
\usepackage{longtable}

\usepackage{helvet}

\usepackage{tabularx}

\renewcommand{\familydefault}{\sfdefault}



%changes default lists style for smaller dashes
\setlist[itemize]{label=--}

%changes the display of chapters, 
%do not displays : "chapter n : name_of_the_chapter"
%displays : "n. name_of_the_chapter" instead
\makeatletter
\def\@makechapterhead#1{%
	\vspace*{0\p@}%
	{	\parindent \z@ \raggedright \normalfont
		\ifnum \c@secnumdepth >\m@ne
		  %\if@mainmatter
		    %\huge\bfseries \@chapapp\space \thechapter
		    \huge\bfseries \thechapter.\space%
		    %\par\nobreak
		    %\vskip 20\p@
		  %\fi
		\fi
		\interlinepenalty\@M
		\huge \bfseries #1\par\nobreak
		\vskip 40\p@
	}}
\makeatother


%adds infos to the document
\title{}

\author{
	Louka Doz
	\and 
	Thibault Meynier
}
\date{\raggedleft \today}


\hypersetup{
	colorlinks=true,
	linkcolor={BlueViolet},
	filecolor=magenta,
	urlcolor={MidnightBlue}
}


%starts to write the document
\begin{document}
	\makeatletter
	\begin{titlepage}

		\begin{flushleft}
			\begin{minipage}{4cm}
				\includegraphics[height=4cm]{img/logo_iut}
			\end{minipage}
			\hfill
			\begin{minipage}{5cm}
				\begin{flushright}
		        	\small IUT Sénart-Fonatinebleau\\
		        	département informatique\\
		        	Route forestière Hurtault\\
		        	77300 Fontainebleau\\
				\end{flushright}
			\end{minipage}
		\end{flushleft}

		\vfill

		\begin{center}
	        \vspace*{3cm}%
	        {\HUGE Ubistock}\\[0.5cm]
	        {\huge Rapport d'avancement hebdomadaire}\\[0.5cm]
	        {\Large \today}\\[0.5cm]
        	{\large \textit{Année scolaire 2019-2020}}\\[1cm]
	    \end{center}
	    \vfill
        \begin{raggedright}
	        \begin{description}
	        	\item[\large \underline{Tutrice de projet :}] \large \textbf{Régine Laleau}\\[1cm]
	        	\item[\underline{Par :}] 
		        	\begin{itemize}
			        		\item Louka Doz
			        		\item Thibault Meynier
			        \end{itemize}
	        \end{description}
        \end{raggedright}        
	    \let\newpage\relax% Avoid following page break
	\end{titlepage}
	\makeatother


	%prints a formated table of contents
	\tableofcontents

	%jumps a page

	\newpage

	\chapter*{Synthèse}
		Durant cette semaine, l’équipe projet a réalisé terminé la réalisation de scripts back-end.
		L’équipe a également élaboré un nouveau système de minimums permettant de fixer une quantité minimale soit sur une ressource, soit sur un ensemble de ressource répondant à un ou plusieurs critères dans un stockage donné. Les critères peuvent être soit un certain nom, soit un certain type, soit les deux.



	 \chapter{Nouvelles implémentations}
	 	\section{Base de données}
	 		\begin{itemize}
	 			\item implémentation du nouveau mécanisme de minimums;
				\item renommage de certains champs pour plus de clarté;
				\item création des tables storage\_minimum et resource\_minimum.
	 		\end{itemize}

		\section{API}
			\begin{itemize}
				\item création/suppression des minimums;
				\item ajout de scripts pour le changement de diverses informations;
				\item mise à jour du script de suppression des stockages pour supprimer les minimums associés et les retirer des groupes auxquels ils appartiennent.

			\end{itemize}

		\section{Interface}
			\begin{itemize}
				\item bouton d’édition des informations sur l’entreprise ajouté;
				\item changement de la mise en forme des liste pour plus de clarté et d’éléments visibles à la fois;
				\item page des membres;
				\item page des logs.

			\end{itemize}

	\chapter{Changements et ajouts}
		\section{Gestion des formulaires}
			Suite à des soucis de répétition de code, la façon dont les formulaires sont validés à été changée. Maintenant, des tags, qui contiennent des morceaux de formulaire (plusieurs champs) sont réutilisable partout dans le programme et se chargent chacun de leur côté de vérifier les champs et d’indiquer à l’utilisateur l’erreur. Le tag parent à juste besoin d’appeler la fonction check() de l’enfant pour connaître la validité des champs que l’enfant possède. La fonction values() des tags enfants renvoie les valeurs des champs.
			Ces problèmes et le temps demandé pour les résoudre, n’a pas permi de réaliser l’ensemble des pages (autres que celle dédiée à l’affichage des ressources et stockages) comme prévu dans le rapport de la semaine dernière.

		\section{Raison des actions pour les logs}
			Le groupe souhaite rajouter un champs, lorsqu’un log est généré, indiquant la raison de l’action. L’idée consiste à permettre à l’utilisateur de justifier son action.
			Le groupe ne sait pas encore si ce nouveau champs sera applicable à toutes les actions ou uniquement aux plus importantes.


	\chapter{objectifs de la prochaine semaine}
		\begin{itemize}
			\item Réaliser des scripts permettant la compatibilité avec les bases de données MySQL;
			\item Terminer l’ensembles des pages de l’interface autre que celle dédiée à l’affichage des ressources et stockages;
			\item Commencer l’ajout des requêtes vers l’API dans les pages existantes.
		\end{itemize}
		
\end{document}