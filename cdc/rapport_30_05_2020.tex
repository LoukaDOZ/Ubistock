%default packages
\documentclass[a4paper, 12pt]{report}
\usepackage{fancyhdr}
\pagestyle{fancyplain}
\fancyhf{}
\rfoot{\thepage}
\lfoot{Louka Doz \& Thibault Meynier}


%adds the ability to put images and figures in the document
\usepackage{graphicx}

%sets default font
\usepackage[T1]{fontenc}

%sets document langage to french
\usepackage[french]{babel}

%adds landscape page orientation
\usepackage{lscape}

%adds easier chapter styles customization
\usepackage{titlesec}

%adds ssmal and HUGE font size
\usepackage[12pt]{moresize}

%for custom list style
\usepackage{enumitem}

%adds links and hyperlinks
\usepackage{hyperref}

%for extra colors
\usepackage[dvipsnames]{xcolor}

%for better tables
\usepackage{array}

%for bigger tables
\usepackage{longtable}

\usepackage{helvet}

\usepackage{tabularx}

\renewcommand{\familydefault}{\sfdefault}



%changes default lists style for smaller dashes
\setlist[itemize]{label=--}

%changes the display of chapters, 
%do not displays : "chapter n : name_of_the_chapter"
%displays : "n. name_of_the_chapter" instead
\makeatletter
\def\@makechapterhead#1{%
	\vspace*{0\p@}%
	{	\parindent \z@ \raggedright \normalfont
		\ifnum \c@secnumdepth >\m@ne
		  %\if@mainmatter
		    %\huge\bfseries \@chapapp\space \thechapter
		    \huge\bfseries \thechapter.\space%
		    %\par\nobreak
		    %\vskip 20\p@
		  %\fi
		\fi
		\interlinepenalty\@M
		\huge \bfseries #1\par\nobreak
		\vskip 40\p@
	}}
\makeatother


%adds infos to the document
\title{}

\author{
	Louka Doz
	\and 
	Thibault Meynier
}
\date{\raggedleft \today}


\hypersetup{
	colorlinks=true,
	linkcolor={BlueViolet},
	filecolor=magenta,
	urlcolor={MidnightBlue}
}


%starts to write the document
\begin{document}
	\makeatletter
	\begin{titlepage}

		\begin{flushleft}
			\begin{minipage}{4cm}
				\includegraphics[height=4cm]{img/logo_iut}
			\end{minipage}
			\hfill
			\begin{minipage}{5cm}
				\begin{flushright}
		        	\small IUT Sénart-Fonatinebleau\\
		        	département informatique\\
		        	Route forestière Hurtault\\
		        	77300 Fontainebleau\\
				\end{flushright}
			\end{minipage}
		\end{flushleft}

		\vfill

		\begin{center}
	        \vspace*{3cm}%
	        {\HUGE Ubistock}\\[0.5cm]
	        {\huge Rapport d'avancement hebdomadaire}\\[0.5cm]
	        {\Large \today}\\[0.5cm]
        	{\large \textit{Année scolaire 2019-2020}}\\[1cm]
	    \end{center}
	    \vfill
        \begin{raggedright}
	        \begin{description}
	        	\item[\large \underline{Tutrice de projet :}] \large \textbf{Régine Laleau}\\[1cm]
	        	\item[\underline{Par :}] 
		        	\begin{itemize}
			        		\item Louka Doz
			        		\item Thibault Meynier
			        \end{itemize}
	        \end{description}
        \end{raggedright}        
	    \let\newpage\relax% Avoid following page break
	\end{titlepage}
	\makeatother


	%prints a formated table of contents
	\tableofcontents

	%jumps a page

	\newpage

	\chapter*{Synthèse}
		Durant cette semaine, l’équipe projet a revu plusieurs éléments d’architecture ce qui a pris du temps et conduira probablement à des changements dans les semaines suivantes.
		L’équipe a longtemps hésité quant à l’implémentation de son système de permissions.


	 \chapter{Nouvelles implémentations}
		\section{Base de données}
			\begin{itemize}
				\item création de script pour mettre en place la base de données à partir d’une base vierge
			\end{itemize}

		\section{API}
			\begin{itemize}
				\item création d’entreprises;
				\item ajout de stockages;
				\item suppression de stockages;
				\item renommage de stockages.
			\end{itemize}

		\section{Interface}
			\begin{itemize}
				\item prototype de page d’accueil
			\end{itemize}

	\chapter{Changements dans le modèle de la base de données}
		\begin{itemize}
			\item Pour ne pas interférer avec le système de gestion de base de données, la table user se nomme maintenant company\_user et son attribut user\_id devient company\_user\_id.
		\end{itemize}
			
			
			

	\chapter{Changements d'architecture}
		\section{Gestion de permissions}
			Les permissions seront accordées aux utilisateurs selon des types/paliers de permissions. 
			Les stockages sur lesquels ils pourront exercer ces permissions seront déterminés par leur appartenance à des groupes.

		\section{Groupes}
			Les groupes se définissent par : un nom et l’appartenance à une entreprise. Les groupes permettent de limiter l’action des utilisateurs sur les stockages et ressources de l’entreprise. Chaque groupe est lié à un ou plusieurs stockages. Les membres du groupe peuvent alors effectuer des actions, en fonction de leur niveau de permission, sur les stockages liés au groupe, ainsi que leurs sous-stockages et ressources.

		\section{Paliers}
			Les paliers fonctionnent un peu comme des niveaux d'accréditation : plus le niveau est haut, plus on a de privilèges. le fondateur de l’entreprise a, bien entendu, le niveau d’accréditation le plus élevé.

			Les privilèges que les utilisateurs possèdent ne s’appliquent qu’au stockages et ressources concernées par le ou les groupes auxquels les utilisateurs appartiennent.
			Voici une première version des droits attribués à chaque palier :

			\begin{description}
				\item [4] visiteur
					\begin{description}
						\item [vision] peut voir la quantité et le nom des ressources + 
						\item [modification] aucun droit
						\item [gestion utilisateur] aucun droit
					\end{description}

				\item [3] collaborateur
					\begin{description}
						\item [vision] palier 4 + voir le reste des informations des resources
						\item [modification] peut augmenter et diminuer la quantité des resources.
						\item [gestion utilisateur] aucun droit
					\end{description}

				\item [2] modérateur
					\begin{description}
						\item [vision] palier 3
						\item [modification] palier 3 + creer, supprimer, renommer et déplacer des stockages ou resources.
						\item [gestion utilisateur] aucun droit
					\end{description}

				\item [1] administrateur (appartient à tous les groupes)
					\begin{description}
						\item [vision] pallier 2
						\item [modification] pallier 2
						\item [gestion utilisateur] créer, supprimer, promouvoir, rétrograder un utilisateur (ne s'applique qu'aux utilisateurs de niveau 2 maximum), affecter un dossier/un utilisateur à groupe, l'en exclure.
					\end{description}

				\item [0] super administrateur apartient à tous les groupes)
					\begin{description}
						\item [vision] palier 1
						\item [modification] palier 1
						\item [gestion utilisateur] palier 1 + peut promouvoir jusqu'au palier 1 + ne peut pas être rétrogradé, supprimé ou exclu d'un groupe par un administrateur
					\end{description}
			\end{description}	


		
\end{document}